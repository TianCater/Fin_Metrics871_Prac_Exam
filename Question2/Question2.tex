\documentclass[11pt,preprint, authoryear]{elsarticle}

\usepackage{lmodern}
%%%% My spacing
\usepackage{setspace}
\setstretch{1.2}
\DeclareMathSizes{12}{14}{10}{10}

% Wrap around which gives all figures included the [H] command, or places it "here". This can be tedious to code in Rmarkdown.
\usepackage{float}
\let\origfigure\figure
\let\endorigfigure\endfigure
\renewenvironment{figure}[1][2] {
    \expandafter\origfigure\expandafter[H]
} {
    \endorigfigure
}

\let\origtable\table
\let\endorigtable\endtable
\renewenvironment{table}[1][2] {
    \expandafter\origtable\expandafter[H]
} {
    \endorigtable
}


\usepackage{ifxetex,ifluatex}
\usepackage{fixltx2e} % provides \textsubscript
\ifnum 0\ifxetex 1\fi\ifluatex 1\fi=0 % if pdftex
  \usepackage[T1]{fontenc}
  \usepackage[utf8]{inputenc}
\else % if luatex or xelatex
  \ifxetex
    \usepackage{mathspec}
    \usepackage{xltxtra,xunicode}
  \else
    \usepackage{fontspec}
  \fi
  \defaultfontfeatures{Mapping=tex-text,Scale=MatchLowercase}
  \newcommand{\euro}{€}
\fi

\usepackage{amssymb, amsmath, amsthm, amsfonts}

\def\bibsection{\section*{References}} %%% Make "References" appear before bibliography


\usepackage[round]{natbib}

\usepackage{longtable}
\usepackage[margin=2.3cm,bottom=2cm,top=2.5cm, includefoot]{geometry}
\usepackage{fancyhdr}
\usepackage[bottom, hang, flushmargin]{footmisc}
\usepackage{graphicx}
\numberwithin{equation}{section}
\numberwithin{figure}{section}
\numberwithin{table}{section}
\setlength{\parindent}{0cm}
\setlength{\parskip}{1.3ex plus 0.5ex minus 0.3ex}
\usepackage{textcomp}
\renewcommand{\headrulewidth}{0.2pt}
\renewcommand{\footrulewidth}{0.3pt}

\usepackage{array}
\newcolumntype{x}[1]{>{\centering\arraybackslash\hspace{0pt}}p{#1}}

%%%%  Remove the "preprint submitted to" part. Don't worry about this either, it just looks better without it:
\makeatletter
\def\ps@pprintTitle{%
  \let\@oddhead\@empty
  \let\@evenhead\@empty
  \let\@oddfoot\@empty
  \let\@evenfoot\@oddfoot
}
\makeatother

 \def\tightlist{} % This allows for subbullets!

\usepackage{hyperref}
\hypersetup{breaklinks=true,
            bookmarks=true,
            colorlinks=true,
            citecolor=blue,
            urlcolor=blue,
            linkcolor=blue,
            pdfborder={0 0 0}}


% The following packages allow huxtable to work:
\usepackage{siunitx}
\usepackage{multirow}
\usepackage{hhline}
\usepackage{calc}
\usepackage{tabularx}
\usepackage{booktabs}
\usepackage{caption}


\newenvironment{columns}[1][]{}{}

\newenvironment{column}[1]{\begin{minipage}{#1}\ignorespaces}{%
\end{minipage}
\ifhmode\unskip\fi
\aftergroup\useignorespacesandallpars}

\def\useignorespacesandallpars#1\ignorespaces\fi{%
#1\fi\ignorespacesandallpars}

\makeatletter
\def\ignorespacesandallpars{%
  \@ifnextchar\par
    {\expandafter\ignorespacesandallpars\@gobble}%
    {}%
}
\makeatother

\newlength{\cslhangindent}
\setlength{\cslhangindent}{1.5em}
\newenvironment{CSLReferences}%
  {\setlength{\parindent}{0pt}%
  \everypar{\setlength{\hangindent}{\cslhangindent}}\ignorespaces}%
  {\par}


\urlstyle{same}  % don't use monospace font for urls
\setlength{\parindent}{0pt}
\setlength{\parskip}{6pt plus 2pt minus 1pt}
\setlength{\emergencystretch}{3em}  % prevent overfull lines
\setcounter{secnumdepth}{5}

%%% Use protect on footnotes to avoid problems with footnotes in titles
\let\rmarkdownfootnote\footnote%
\def\footnote{\protect\rmarkdownfootnote}
\IfFileExists{upquote.sty}{\usepackage{upquote}}{}

%%% Include extra packages specified by user
\usepackage{booktabs}
\usepackage{longtable}
\usepackage{array}
\usepackage{multirow}
\usepackage{wrapfig}
\usepackage{float}
\usepackage{colortbl}
\usepackage{pdflscape}
\usepackage{tabu}
\usepackage{threeparttable}
\usepackage{threeparttablex}
\usepackage[normalem]{ulem}
\usepackage{makecell}
\usepackage{xcolor}

%%% Hard setting column skips for reports - this ensures greater consistency and control over the length settings in the document.
%% page layout
%% paragraphs
\setlength{\baselineskip}{12pt plus 0pt minus 0pt}
\setlength{\parskip}{12pt plus 0pt minus 0pt}
\setlength{\parindent}{0pt plus 0pt minus 0pt}
%% floats
\setlength{\floatsep}{12pt plus 0 pt minus 0pt}
\setlength{\textfloatsep}{20pt plus 0pt minus 0pt}
\setlength{\intextsep}{14pt plus 0pt minus 0pt}
\setlength{\dbltextfloatsep}{20pt plus 0pt minus 0pt}
\setlength{\dblfloatsep}{14pt plus 0pt minus 0pt}
%% maths
\setlength{\abovedisplayskip}{12pt plus 0pt minus 0pt}
\setlength{\belowdisplayskip}{12pt plus 0pt minus 0pt}
%% lists
\setlength{\topsep}{10pt plus 0pt minus 0pt}
\setlength{\partopsep}{3pt plus 0pt minus 0pt}
\setlength{\itemsep}{5pt plus 0pt minus 0pt}
\setlength{\labelsep}{8mm plus 0mm minus 0mm}
\setlength{\parsep}{\the\parskip}
\setlength{\listparindent}{\the\parindent}
%% verbatim
\setlength{\fboxsep}{5pt plus 0pt minus 0pt}



\begin{document}



\begin{frontmatter}  %

\title{Financial Econometrics 871 Practical Exam 2022: Question 2}

% Set to FALSE if wanting to remove title (for submission)




\author[Add1]{Tian Cater}
\ead{19025831@sun.ac.za}





\address[Add1]{University of Stellenbosch, Western Cape, South
Africa\footnote{2022-11-27}}


\begin{abstract}
\small{
Economists recently pointed out that the current yield spreads in SA mid
to longer-dated bond yields have since 2020 been the highest in decades.
I analyse the current yield spreads in the local bond market for this
question. In addition, I compare the local spread to comparable
international spreads, observe the correlation between the local bond
spreads and the USD-ZAR level, and consider the SA 10 Year Break-Even
inflation estimate. The findings suggest that the current yield spreads
in SA mid to longer-dated bond yields have been the highest in decades
since 2020 and even 2018. Additionally, the strong co-movement between
the SA bond yield spread and the ZAR/USD exchange rate has remained
strong; however, it marginally diminished between 2010 and 2016. Lastly,
the SA average inflation rate has not surpassed the BE inflation yield
estimate since 2014, indicating that fixed investment has been firmly
preferred over index-linked bond investment.
}
\end{abstract}

\vspace{1cm}





\vspace{0.5cm}

\end{frontmatter}



%________________________
% Header and Footers
%%%%%%%%%%%%%%%%%%%%%%%%%%%%%%%%%
\pagestyle{fancy}
\chead{}
\rhead{Question 2}
\lfoot{}
\rfoot{\footnotesize Page \thepage}
\lhead{}
%\rfoot{\footnotesize Page \thepage } % "e.g. Page 2"
\cfoot{}

%\setlength\headheight{30pt}
%%%%%%%%%%%%%%%%%%%%%%%%%%%%%%%%%
%________________________

\headsep 35pt % So that header does not go over title




\hypertarget{comparing-global-bond-market-yield-spreads}{%
\section{Comparing Global Bond Market Yield
Spreads}\label{comparing-global-bond-market-yield-spreads}}

Figure \ref{Figure1} shows the bond market yield spreads for SA, the US,
Brazil, Germany, China, and Japan, respectively. Visually, \ref{Figure1}
confirms the notion that the current yield spreads in local mid to
longer-dated bond yields have since 2020 been the highest in decades.
The SA Bond yield spread has been significantly more volatile than those
of Germany, China, the US, and Japan while having less volatility in the
last few years than Brazil.

\begin{figure}[H]

{\centering \includegraphics{Question2_files/figure-latex/Global_bonds_plot-1} 

}

\caption{Global Bond Market Yield Spreads \label{Figure1}}\label{fig:Global_bonds_plot}
\end{figure}

In considering the SA and US bond yield spreads against the ZAR/USD
exchange rate, graphed in \ref{Figure2} below, it is evident that the
co-movements between the ZAR/USD and the SA bonds yield spread have over
time been relatively strong, however, between the periods of 2010 and
2016 this positive correlation has diminished, likely due to the large
open market asset purchases by the FED that distorts the international
spillovers.

\begin{figure}[H]

{\centering \includegraphics{Question2_files/figure-latex/SA_Bonds_Plot-1} 

}

\caption{SA Bond Yields, Spread, and ZAR/USD Exchange Rate \label{Figure2}}\label{fig:SA_Bonds_Plot}
\end{figure}

To provide an additional perspective, the statistics in Table
\ref{Table1} compared the SA and US bond yield spread before and after
the Global Financial Crises (GFC). Overall, considering the SA and US
separately and somewhat loosely, as done here, the volatility in the SA
bond yield has remained relatively similar compared to its pre- and
post-GFC.

\begin{table}

\caption{\label{tab:stats table}SA vs US Bond Yield Spread Statistics: Pre GFC vs Post GFC. \label{Table1}}
\centering
\begin{tabular}[t]{lrrrr}
\toprule
\multicolumn{1}{c}{ } & \multicolumn{2}{c}{Pre GFC} & \multicolumn{2}{c}{Post GFC} \\
\cmidrule(l{3pt}r{3pt}){2-3} \cmidrule(l{3pt}r{3pt}){4-5}
Description & SA & US & SA & US\\
\midrule
Minimum & -0.51 & -1.80 & -0.05 & -0.58\\
Arithmetic Mean & 1.04 & 0.50 & 1.43 & 1.97\\
Maximum & 2.74 & 2.85 & 2.91 & 6.30\\
SE Mean & 0.02 & 0.02 & 0.01 & 0.02\\
LCL Mean (0.95) & 1.00 & 0.46 & 1.40 & 1.93\\
\addlinespace
UCL Mean (0.95) & 1.08 & 0.55 & 1.45 & 2.00\\
Stdev & 0.96 & 1.10 & 0.80 & 0.99\\
Skewness & 0.04 & -0.14 & 0.03 & 1.15\\
Kurtosis & -1.52 & -0.92 & -1.10 & 1.39\\
\bottomrule
\end{tabular}
\end{table}

\hypertarget{sa-break-even-inflation}{%
\section{SA Break-Even Inflation}\label{sa-break-even-inflation}}

Lastly, I analyse the SA Break-Even inflation yield estimate and compare
it to the SA inflation rate (Figure \ref{Figure4}). Break-even inflation
is the difference between the nominal yield on a fixed-rate investment
and the real yield (fixed spread) on an inflation-linked investment of
similar maturity and credit quality. If inflation averages more than the
break-even, the inflation-linked investment will outperform the fixed
rate.

From Figure \ref{Figure4}, the SA average inflation rate has not
surpassed the BE inflation yield estimate since 2014, indicating that
fixed investment has been firmly preferred over index-linked bond
investment.

\begin{figure}[H]

{\centering \includegraphics{Question2_files/figure-latex/BEI_infl_plot-1} 

}

\caption{SA Break-Even Inflation Yield Versus Average Inflation Rate \label{Figure4}}\label{fig:BEI_infl_plot}
\end{figure}

\newpage

\hypertarget{references}{%
\section*{References}\label{references}}
\addcontentsline{toc}{section}{References}

\hypertarget{refs}{}
\begin{CSLReferences}{0}{0}
\end{CSLReferences}

\bibliography{Tex/ref}





\end{document}
