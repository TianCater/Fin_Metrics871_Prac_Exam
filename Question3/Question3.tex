\documentclass[11pt,preprint, authoryear]{elsarticle}

\usepackage{lmodern}
%%%% My spacing
\usepackage{setspace}
\setstretch{1.2}
\DeclareMathSizes{12}{14}{10}{10}

% Wrap around which gives all figures included the [H] command, or places it "here". This can be tedious to code in Rmarkdown.
\usepackage{float}
\let\origfigure\figure
\let\endorigfigure\endfigure
\renewenvironment{figure}[1][2] {
    \expandafter\origfigure\expandafter[H]
} {
    \endorigfigure
}

\let\origtable\table
\let\endorigtable\endtable
\renewenvironment{table}[1][2] {
    \expandafter\origtable\expandafter[H]
} {
    \endorigtable
}


\usepackage{ifxetex,ifluatex}
\usepackage{fixltx2e} % provides \textsubscript
\ifnum 0\ifxetex 1\fi\ifluatex 1\fi=0 % if pdftex
  \usepackage[T1]{fontenc}
  \usepackage[utf8]{inputenc}
\else % if luatex or xelatex
  \ifxetex
    \usepackage{mathspec}
    \usepackage{xltxtra,xunicode}
  \else
    \usepackage{fontspec}
  \fi
  \defaultfontfeatures{Mapping=tex-text,Scale=MatchLowercase}
  \newcommand{\euro}{€}
\fi

\usepackage{amssymb, amsmath, amsthm, amsfonts}

\def\bibsection{\section*{References}} %%% Make "References" appear before bibliography


\usepackage[round]{natbib}

\usepackage{longtable}
\usepackage[margin=2.3cm,bottom=2cm,top=2.5cm, includefoot]{geometry}
\usepackage{fancyhdr}
\usepackage[bottom, hang, flushmargin]{footmisc}
\usepackage{graphicx}
\numberwithin{equation}{section}
\numberwithin{figure}{section}
\numberwithin{table}{section}
\setlength{\parindent}{0cm}
\setlength{\parskip}{1.3ex plus 0.5ex minus 0.3ex}
\usepackage{textcomp}
\renewcommand{\headrulewidth}{0.2pt}
\renewcommand{\footrulewidth}{0.3pt}

\usepackage{array}
\newcolumntype{x}[1]{>{\centering\arraybackslash\hspace{0pt}}p{#1}}

%%%%  Remove the "preprint submitted to" part. Don't worry about this either, it just looks better without it:
\makeatletter
\def\ps@pprintTitle{%
  \let\@oddhead\@empty
  \let\@evenhead\@empty
  \let\@oddfoot\@empty
  \let\@evenfoot\@oddfoot
}
\makeatother

 \def\tightlist{} % This allows for subbullets!

\usepackage{hyperref}
\hypersetup{breaklinks=true,
            bookmarks=true,
            colorlinks=true,
            citecolor=blue,
            urlcolor=blue,
            linkcolor=blue,
            pdfborder={0 0 0}}


% The following packages allow huxtable to work:
\usepackage{siunitx}
\usepackage{multirow}
\usepackage{hhline}
\usepackage{calc}
\usepackage{tabularx}
\usepackage{booktabs}
\usepackage{caption}


\newenvironment{columns}[1][]{}{}

\newenvironment{column}[1]{\begin{minipage}{#1}\ignorespaces}{%
\end{minipage}
\ifhmode\unskip\fi
\aftergroup\useignorespacesandallpars}

\def\useignorespacesandallpars#1\ignorespaces\fi{%
#1\fi\ignorespacesandallpars}

\makeatletter
\def\ignorespacesandallpars{%
  \@ifnextchar\par
    {\expandafter\ignorespacesandallpars\@gobble}%
    {}%
}
\makeatother

\newlength{\cslhangindent}
\setlength{\cslhangindent}{1.5em}
\newenvironment{CSLReferences}%
  {\setlength{\parindent}{0pt}%
  \everypar{\setlength{\hangindent}{\cslhangindent}}\ignorespaces}%
  {\par}


\urlstyle{same}  % don't use monospace font for urls
\setlength{\parindent}{0pt}
\setlength{\parskip}{6pt plus 2pt minus 1pt}
\setlength{\emergencystretch}{3em}  % prevent overfull lines
\setcounter{secnumdepth}{5}

%%% Use protect on footnotes to avoid problems with footnotes in titles
\let\rmarkdownfootnote\footnote%
\def\footnote{\protect\rmarkdownfootnote}
\IfFileExists{upquote.sty}{\usepackage{upquote}}{}

%%% Include extra packages specified by user
\usepackage{booktabs}
\usepackage{longtable}
\usepackage{array}
\usepackage{multirow}
\usepackage{wrapfig}
\usepackage{float}
\usepackage{colortbl}
\usepackage{pdflscape}
\usepackage{tabu}
\usepackage{threeparttable}
\usepackage{threeparttablex}
\usepackage[normalem]{ulem}
\usepackage{makecell}
\usepackage{xcolor}

%%% Hard setting column skips for reports - this ensures greater consistency and control over the length settings in the document.
%% page layout
%% paragraphs
\setlength{\baselineskip}{12pt plus 0pt minus 0pt}
\setlength{\parskip}{12pt plus 0pt minus 0pt}
\setlength{\parindent}{0pt plus 0pt minus 0pt}
%% floats
\setlength{\floatsep}{12pt plus 0 pt minus 0pt}
\setlength{\textfloatsep}{20pt plus 0pt minus 0pt}
\setlength{\intextsep}{14pt plus 0pt minus 0pt}
\setlength{\dbltextfloatsep}{20pt plus 0pt minus 0pt}
\setlength{\dblfloatsep}{14pt plus 0pt minus 0pt}
%% maths
\setlength{\abovedisplayskip}{12pt plus 0pt minus 0pt}
\setlength{\belowdisplayskip}{12pt plus 0pt minus 0pt}
%% lists
\setlength{\topsep}{10pt plus 0pt minus 0pt}
\setlength{\partopsep}{3pt plus 0pt minus 0pt}
\setlength{\itemsep}{5pt plus 0pt minus 0pt}
\setlength{\labelsep}{8mm plus 0mm minus 0mm}
\setlength{\parsep}{\the\parskip}
\setlength{\listparindent}{\the\parindent}
%% verbatim
\setlength{\fboxsep}{5pt plus 0pt minus 0pt}



\begin{document}



\begin{frontmatter}  %

\title{Financial Econometrics 871 Practical Exam 2022: Question 3}

% Set to FALSE if wanting to remove title (for submission)




\author[Add1]{Tian Cater}
\ead{19025831@sun.ac.za}





\address[Add1]{University of Stellenbosch, Western Cape, South
Africa\footnote{2022-11-27}}


\begin{abstract}
\small{
To answer this question, I use the information on the ALSI (J200) and
SWIX (J400) top 40 Indexes and compare the SWIX and ALSI methodologies
by looking at the performance of different sector exposures and stock
concentrations over time. The findings show that the (uncapped) ALSI has
performed relatively better since the onset of COVID-19 than the SWIX
and can be attributed to a higher weight assigned to the
better-performing Resources sector. In addition, through analysing the
impact that 6\% and 10\% capping levels would have had on both the SWIX
and ALSI, it is apparent that a 6\% cap on the ALSI would have
significantly reduced its performance in the past few years due to
slicing the weights on the higher returning Resources sector. On the
other hand, the 10\% cap on SWIX would have improved its return since
the onset of COVID, likely due to a reduction in weighted contribution
from the weaker-performing Industrial and Financial sectors.
}
\end{abstract}

\vspace{1cm}





\vspace{0.5cm}

\end{frontmatter}



%________________________
% Header and Footers
%%%%%%%%%%%%%%%%%%%%%%%%%%%%%%%%%
\pagestyle{fancy}
\chead{}
\rhead{Question 3}
\lfoot{}
\rfoot{\footnotesize Page \thepage}
\lhead{}
%\rfoot{\footnotesize Page \thepage } % "e.g. Page 2"
\cfoot{}

%\setlength\headheight{30pt}
%%%%%%%%%%%%%%%%%%%%%%%%%%%%%%%%%
%________________________

\headsep 35pt % So that header does not go over title




\hypertarget{alsi-and-swix-uncapped-weighted-portfolio-returns}{%
\section{ALSI and SWIX (uncapped) Weighted Portfolio
Returns}\label{alsi-and-swix-uncapped-weighted-portfolio-returns}}

Figure \ref{Figure1} plots the weighted (uncapped) cumulative returns
series for the ALSI and SWIX indexes. The cumulative weighted returns
for the ALSI and SWIX indexes are strikingly similar, however, since the
onset of COVID-19, the ALSI has achieved higher returns.

\begin{figure}[H]

{\centering \includegraphics{Question3_files/figure-latex/Indexes_Cum_ret_plot-1} 

}

\caption{Cumulative Returns of ALSI and SWIX Indexes Per Sector. \label{Figure1}}\label{fig:Indexes_Cum_ret_plot}
\end{figure}

\hypertarget{weighted-return-contribition-per-sector}{%
\section{Weighted Return Contribition per
Sector}\label{weighted-return-contribition-per-sector}}

In order to shed some light on SWIX's relatively weaker performance
since the onset of COVID, I analyse the dynamic weighted contribution of
of each sector to the indexes. Figures \ref{Figure2} and \ref{Figure3}
shows the dynamic weight contribution to the ALSI and SWIX portfolios,
respectively.

In considering the ALSI in Figure \ref{Figure2}, a capping of 10\% has
virtually the same cumulative returns as when uncapped, with
significantly lower returns when capped at 6\%. This is likely due to
this 6\% capping most of the relatively higher returns generated from
the Resources sector. On the other hand, from the SWIX in Figure
\ref{Figure3}, the 10\% cap would have improved its return since the
onset of COVID, most likely due to a reduction in weighted contribution
from the weaker performing Industrial and Financial sectors.

\begin{figure}[H]

{\centering \includegraphics{Question3_files/figure-latex/ALSI weights plot-1} 

}

\caption{Dynamic Weight Contribution to ALSI Portfolio Per Sector. \label{Figure2}}\label{fig:ALSI weights plot}
\end{figure}

\begin{figure}[H]

{\centering \includegraphics{Question3_files/figure-latex/SWIX weights plot-1} 

}

\caption{Dynamic Weight Contribution to SWIX Portfolio. \label{Figure3}}\label{fig:SWIX weights plot}
\end{figure}

Figures \ref{Figure4} and \ref{Figure5} confirms the analyses above.

\begin{figure}[H]

{\centering \includegraphics{Question3_files/figure-latex/ALSI contr plot-1} 

}

\caption{Dynamic Weighted Return Contribution to SWIX Portfolio Per Sector. \label{Figure4}}\label{fig:ALSI contr plot}
\end{figure}

\begin{figure}[H]

{\centering \includegraphics{Question3_files/figure-latex/SWIX contr plot-1} 

}

\caption{Dynamic Weighted Return Contribution to SWIX Portfolio Per Sector. \label{Figure5}}\label{fig:SWIX contr plot}
\end{figure}

\hypertarget{counterfactual-6-and-10-capped-alsi-and-swix-returns}{%
\section{Counterfactual 6\% and 10\% Capped ALSI and SWIX
Returns}\label{counterfactual-6-and-10-capped-alsi-and-swix-returns}}

Figures \ref{Figure6} and \ref{Figure7} shows the counterfactual impact
that 6\% and 10\% capping levels would have had on both the SWIX and
ALSI, respectively. In investigating these figures, it is apparent that
a 6\% cap on the ALSI would have significantly reduced its performance
in the past few years due to slicing the weights on the higher returning
Resources sector. On the other hand, the 10\% cap on SWIX would have
improved its return since the onset of COVID, likely due to a reduction
in weighted contribution from the weaker-performing Industrial and
Financial sectors.

\begin{figure}[H]

{\centering \includegraphics{Question3_files/figure-latex/ALSI capping plot-1} 

}

\caption{Cumulative Returns of ALSI Capped at 6 and 10 Percent. \label{Figure6}}\label{fig:ALSI capping plot}
\end{figure}

\begin{figure}[H]

{\centering \includegraphics{Question3_files/figure-latex/SWIX capping plot-1} 

}

\caption{Cumulative Returns of SWIX Capped at 6 and 10 Percent. \label{Figure7}}\label{fig:SWIX capping plot}
\end{figure}

\newpage

\hypertarget{references}{%
\section*{References}\label{references}}
\addcontentsline{toc}{section}{References}

\hypertarget{refs}{}
\begin{CSLReferences}{0}{0}
\end{CSLReferences}

\bibliography{Tex/ref}





\end{document}
